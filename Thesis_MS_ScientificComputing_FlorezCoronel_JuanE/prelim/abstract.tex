%Hi! We encourage you to visit https://libguides.uprm.edu/writingclinics and check out the \textbf{Abstracts Clinic.} Keep in mind that depending on your discipline, abstracts should be a \textbf{single paragraph}, at around \textbf{200-400 words}. It should concisely but clearly summarize your thesis document. The \textbf{IMRaD format} is recommended for writing abstracts: Introduction (1-3 sentences long, present tense), Methodology (1-3 sentences long, past tense), Results (1-3 sentences long, past tense), and Discussion (1-2 sentences long, present tense). Remember that the number of sentences and verb tense are only guidelines!
\vspace*{0.5in}
\begin{center}
\section*{ABSTRACT}
\end{center}
\addcontentsline{toc}{section}{ABSTRACT} 

\acrfull{fmri} is a widely used non-invasive medical procedure for studying 
brain function. Identifying activated regions of the brain is a common challenge 
in \acrshort{fmri} analysis. Low-signal and small data cases pose significant 
difficulties for activation detection. These scenarios arise when studying 
high-level cognitive tasks or single-subject experiments, respectively. In 
this study, we propose an innovative algorithm, entitled \acrfull{bfast}, 
which utilizes smoothing and extreme value theory on probabilistic maps to 
find threshold values. The algorithm's performance was evaluated on artificial 
data that simulated a range of signal magnitudes. The results were promising, 
with an average similarity of 85\% with respect to the expected output. 
Furthermore, the proposed procedure was applied to a study that aimed to 
identify the cerebral regions responsible for processing beliefs and questions 
as stimuli. Our findings suggest that the \acrshort{bfast} algorithm holds 
promise for detecting activated areas in the brain with high accuracy, 
particularly in cases involving low-signal and small data. Such advancements 
in \acrshort{fmri} analysis algorithms could lead to more accurate and precise 
studies of brain function, with significant implications for both clinical and 
research settings.

\newpage

\vspace*{0.5in}
\begin{center}
\section*{RESUMEN}
\end{center}
\addcontentsline{toc}{section}{RESUMEN}

La técnica de Imágenes por Resonancia Magnética Funcional (\acrshort{fmri}) 
es un procedimiento médico no invasivo ampliamente utilizado para estudiar 
la función cerebral. Identificar regiones activadas del cerebro es un desafío 
común en el análisis de \acrshort{fmri}. Los casos de baja señal y datos 
pequeños plantean desafíos importantes para la detección de activación. 
Estos escenarios surgen cuando se estudian tareas cognitivas de alto nivel o 
experimentos con un solo sujeto, respectivamente. En este estudio, proponemos 
un algoritmo innovador, titulado Umbralizado y Suavizado Adaptativo 
Bayesiano Rápido (\acrshort{bfast}), que utiliza la teoría de suavizado 
y valores extremos en mapas probabilísticos para encontrar valores de 
umbral. El rendimiento del algoritmo se evaluó a partir de datos artificiales 
que simularon una variedad de magnitudes de señal. Los resultados fueron 
prometedores, con una similitud promedio del 85\% con respecto al resultado 
esperado. Además, el procedimiento propuesto se aplicó a un estudio que tenía 
como objetivo identificar las regiones cerebrales responsables de procesar 
creencias y preguntas como estímulos. Nuestros hallazgos sugieren que el 
algoritmo \acrshort{bfast} es prometedor para detectar regiones activadas 
en el cerebro con alta precisión, particularmente en casos que involucran 
baja señal y datos pequeños. Estos avances en los algoritmos de análisis 
de \acrshort{fmri} podrían conducir a estudios más exactos y precisos de la 
función cerebral, con importantes implicaciones tanto para entornos clínicos 
como de investigación.