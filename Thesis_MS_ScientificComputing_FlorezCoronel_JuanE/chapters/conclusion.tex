\chapter{Conclusions}

In this thesis, we've introduced a new approach for detecting brain activity 
in single subject \gls{fmri} data using adaptive smoothing and 
thresholding alongside with some Bayesian analysis. Our work has met its 
main goals, providing valuable 
insights and tools for the analysis of neuroimaging data.

We started by performing a Bayesian time-series analysis to create a 
posterior probability map for single-subject \gls{fmri} images. This bayesian 
approach handles the \gls{fmri} data in a very effective and structured manner 
such that the resulting probability maps present a reliable image 
of potential brain activations. Next, we developed an adaptive smoothing and 
thresholding algorithm called \gls{bfast} to process these probability maps and 
identify the active brain regions. By dynamically adjusting the probabilities 
given their spatial distribution, our method improves the detection of true 
activations while reducing noise. This adaptive approach ensures that the true 
active voxels are detected at each step and errors due to noise is reduced, enhancing 
the overall accuracy and reliability of the results.

We tested our algorithm extensively through various simulation scenarios. 
The results showed that our method performs well in terms of similarity 
measures, false positive rates, and the percentage of detected activations. 
These simulations demonstrated that our approach is effective, 
especially in challenging conditions with low \gls{snr} values. Finally, we applied our algorithm to a real \gls{fmri} dataset, which confirmed 
its practical utility. The real-world application showed that our method 
can detect brain activations that align with established patterns 
of brain activity. 

In summary, this thesis has developed and validated a new bayesian adaptive 
smoothing and thresholding method for detecting brain activity in single-subject 
\gls{fmri} data. Our contributions include a bayesian analysis framework, 
an adaptive processing method, and thorough testing through simulations and 
real data.

%Future Works

Modify parameters from the \gls{bfast} algorithm such as the
normalization coefficients, the computation of the threshold,
and the smoothing coefficient, among others.

Compare the results obtained by the \gls{bfast} algorithm in real life
applications with the results obtained by equivalent methods.