\chapter{Literature Review}  

\section{Introduction to \texorpdfstring{\gls{fmri}}{fMRI}}

\gls{mri} is a powerful medical imaging technique that has revolutionized the field of diagnostic medicine \cite{westbrook2018mri}. At its core, \gls{mri} relies on the interaction of protons within the human body with strong magnetic fields and radiofrequency pulses \cite{hashemi2012mri}. These magnetic fields, often generated by superconducting magnets, align the protons within the body's tissues \cite{berger2002does}. Subsequent radiofrequency pulses perturb this alignment, causing the protons to emit radiofrequency signals as they return to their original alignment. By detecting these signals and their variations, \gls{mri} scanners create high-resolution anatomical images that provide detailed insights into the body's internal structures \cite{guven2023brain}. This non-invasive and versatile imaging modality has become indispensable in clinical diagnosis, research, and medical practice, offering a wealth of information for assessing various medical conditions.

As traditional \gls{mri} focuses mainly on the generation of static anatomic images of the internal structures of the body \cite{westbrook2018mri}, \gls{fmri} brings a new advantage as it captures the dynamic activities of the body part studied \cite{buchbinder2016functional, christopher2008applications, logothetis2008we}. The critical difference is that magnetic resonance is based mainly on the interaction of protons with magnetic fields to produce detailed anatomic images in the data acquisition process. At the same time, the \gls{fmri} takes advantage of the \gls{bold} contrast to indirectly measure neural activity by detecting changes in the oxygenation level of the blood \cite{smith2004overview}. This fundamental change of emphasis allows \gls{fmri} to study the visualization and mapping of brain regions activated during specific cognitive tasks, making it a very used tool in cognitive neuroscience and neuropsychology \cite{orchard2003simultaneous, ruschemeyer2006native}. 

In the domain of \gls{fmri}, \gls{bold} contrast is within the most important concepts to be studied \cite{logothetis2004nature}. The essence of \gls{bold} contrast relies on the observation that neural activity generates changes in local blood oxygenation levels \cite{lindquist2008rapid}. As brain regions become more active, they demand increased oxygen and glucose to sustain their functions \cite{lindquist2008statistical}. In response, blood flow to these regions is expected to be altered to meet the demand. Importantly, hemoglobin, the oxygen-carrying molecule in blood, behaves differently when oxygenated and deoxygenated, affecting its magnetic properties \cite{uyuklu2009effect, pauling1936magnetic, bren2015discovery}. When oxygenation levels of the blood change, it generates fluctuations in its magnetic properties; this process is all captured by \gls{fmri} experiments \cite{buxton2012dynamic}.

As expected, this long data reading process generates a significant amount of noise because of all the factors that are expected to work correctly during the measurements. In addition to that, it is known that high-level cognitive tasks produce low-signal scenarios in \gls{fmri} experiments \cite{cui2011quantitative}. To quantify the amount of noise concerning the signal studied, researchers use metrics such as the \gls{snr} and the \gls{cnr} \cite{welvaert2013definition}. The \gls{snr} quantifies the ratio of the strength of the signal arising from brain activity to the background noise inherent in the imaging process. Higher \gls{snr} values indicate a more robust and detectable signal. Similarly, \gls{cnr} assesses the contrast between activated and non-activated brain regions by comparing the difference in signal intensity between them to the noise level. A higher \gls{cnr} signifies a stronger and more discernible activation signal relative to background noise.

\section{Analysis of \texorpdfstring{\gls{fmri}}{fMRI} Data - Time Series, Activation Detection and Final Image}

Voxels, short for volumetric pixels, are fundamental building blocks in \gls{fmri} analysis \cite{norman2006beyond}. They represent \gls{3d} units within the image and play a crucial role in discretizing the space studied. Each voxel corresponds to a tiny, well-defined volume in the brain, and within this volume, \gls{fmri} data, particularly \gls{bold} signal measurements, are collected over time \cite{li2009voxel}. These measurements over time can be compiled into a time series and, more specifically, with a linear relation. The \gls{glm} for time series analysis is a fundamental technique in \gls{fmri} data processing because it captures temporal dynamics of neural activity \cite{kiebel2007general, friston1994statistical}.

Detection of neural activity in the \gls{roi} is a crucial field of study in \gls{fmri} research as it enables scientists to identify brain regions that exhibit significant changes in activity in response to specific stimuli or tasks \cite{ardekani1999activation}. The identification of the \gls{roi} in \gls{fmri} is called image masking \cite{peer2016intensity}. The \gls{roi} can correspond to anatomically defined brain structures, functionally significant areas, or areas of interest for a particular study \cite{poldrack2007region}. Image masking is employed to improve the precision and efficiency of analyses, as it allows researchers to isolate and concentrate on the neural activity occurring within predefined brain regions \cite{mitsis2008regions}. By delineating the \gls{roi}, image masking effectively filters out irrelevant data, reducing noise and enhancing the sensitivity of statistical analyses. One method to apply the image masking, as implemented in NiLearn \cite{abraham2014machine}, is based on a heuristic proposed by T.Nichols \cite{luo2003diagnosis}: find the least dense point of the histogram, between a lower cutoff and an upper cutoff of the total image histogram.

Within the area of neural activity, some researchers use frequentist approaches to detect activation in \gls{fmri} studies. These approaches can be described as statistical methods that adopt a null hypothesis tested using p-values to determine whether a brain region is significantly activated by a particular stimulus or condition \cite{friston2002classical, almodovar2019fast}. These methods are widely used in \gls{fmri} research \cite{almodovar2019fast, josephs1997event, worsley1995analysis, worsley1996searching}. Still, they have been criticized for their limitations, such as their problems addressing hemodynamic variability and the spatio-temporal autocorrelations in \gls{fmri} \cite{woolrich2012years}.

An essential tool to be discussed that is relevant in generating low-noise activation maps is image smoothing \cite{lee1983digital}. Image smoothing is a crucial step in activation map analysis because it helps to reduce noise and improve the localization of activated brain regions \cite{lindquist2010adaptive, strappini2017adaptive, garg2016quality}. By smoothing the probability maps, researchers can more easily identify the brain regions most strongly activated by a particular stimulus or condition \cite{tabelow2006analyzing}. Adaptive smoothing has been a common technique in activation detection in \gls{fmri}, and researchers have always complimented this technique with frequentist approaches \cite{triantafyllou2006effect, mikl2008effects, liu2017functional}. These methods yield precise results. However, there is a gap in the literature regarding using adaptive smoothing with Bayesian approaches.

After obtaining the final activation map, researchers must be able to compare methods and test their findings' reliability. Hence, tools like the \gls{ji} were introduced to the area of \gls{fmri}. The \gls{ji} was initially introduced by Paul Jaccard in 1901 \cite{jaccard1901etude}; later, researchers found application in \gls{fmri} analysis, as discussed in \cite{maitra2010re}. The \gls{ji} measures the similarity between two sets by calculating the intersection over the union of their elements. In \gls{fmri}, it assesses the overlap and consistency of brain activation patterns across different subjects, conditions, or studies. A higher \gls{ji} indicates a more remarkable similarity between activation maps.

\section{Bayesian Analysis}

Bayesian analysis is essential in data analysis and statistical reasoning \cite{bernardo1994bayesian, bolstad2016introduction}. It is a probabilistic framework that quantifies uncertainty and makes inferences from data \cite{van2021bayesian}. Unlike traditional frequentist statistics, which treat model parameters as fixed and unknown values, Bayesians treat these parameters as random variables, encapsulating our uncertainty about their values with probability distributions \cite{bayarri2004interplay}. In the Bayesian analysis, prior beliefs about parameters are combined with observed data through Bayes' Rule to construct the posterior distribution. The prior distribution is the key in Bayesian approaches as it represents the previous knowledge or assumptions about the random variables in question \cite{gelman2002prior, stone2013bayes}.

The selection of a prior distribution is relevant in Bayesian modeling, as it profoundly influences the posterior distribution \cite{kass1996formal}. When choosing a prior distribution, researchers must balance incorporating relevant domain expertise and ensuring that the prior does not dominate the outcome of the posterior. This requires careful consideration of the prior's shape, scale, and informativeness \cite{perez2002expected}. Various methods, such as non-informative or weakly informative priors, hierarchical modeling, and empirical Bayes techniques, offer strategies for selecting appropriate priors based on the available information and the specific context of the analysis \cite{terenin2017noninformative}. A good choice of prior distributions incorporates valuable previous knowledge while preserving the capacity of data to update and refine the result, thus yielding more robust and insightful posterior distributions \cite{gelman2013bayesian}.

\section{Relevant Distributions}

Given the nature of random variables that represent probability values, distributions whose range lies between 0 and 1 are studied. The \gls{tn} is a relevant probability distribution with applications in modeling extreme values that fall within a specific range \cite{burkardt2014truncated}. This distribution is characterized by the constraint that its values lie within a defined interval, effectively truncating the tails of the standard normal distribution.

In \cite{david2004order}, the concept of the \gls{dma} is presented as a fundamental idea of the \gls{evt}. The \gls{dma} characterizes the asymptotic behavior of extreme value distributions as it represents a specific class of distributions that exhibit remarkable convergence properties when dealing with extreme values. It is the set of distributions for which the maxima of independent and identically distributed random variables converge to one of the three extreme value distributions: the Gumbel, Fréchet, or Weibull distribution, depending on the characteristics of the underlying distribution \cite{gorgoso2014use}.