\chapter{Methodology}

\section{Time-Series Model}

A time series analysis for each voxel of the image will allow temporal fluctuations in \gls{bold} signals to be captured. By tracking changes in \gls{bold} signals over time, the objective is to investigate dynamic patterns of brain activity, allowing for the identification of regions that respond to specific stimuli or tasks. Analyzing time series data at the voxel level provides valuable information into the temporal dynamics of neural processes, enabling a deep understanding of the brain's architecture.

Let $\bm{y}_i$ be a vector of the response variable of the $ith$ voxel, and $\bm{X}$ be the design matrix of the study containing the expected \gls{bold} and orthogonal drift components to take account of the low-frequency effects during the reading. With $\bm{\beta}_i$ being the vector of coefficients associated with the stimulus, we will have $\bm{y}_i \sim N ( \bm{X} \bm{\beta}_i, \bm{\Sigma})$. Note that $\bm{\Sigma}$ can have a \gls{arma} structure. However, if we let $\bm{\Sigma}=\sigma^2 \bm{I}$, the independent model is obtained: 
$
\bm{y}_i|\bm{\beta}_i, \sigma, \bm{X} \sim N \left(\bm{X} \bm{\beta}_i,\sigma^2 \bm{I}\right).
$

% Change pi to N in the prior?
From here, the procedure is presented in \cite{gelman2013bayesian} explains that to obtain the posterior distribution of the coefficients associated with the stimulus, $\bm{\beta}_i$. We will use a noninformative prior distribution that is uniform on $(\bm{\beta}_i,\log \sigma)$:

\begin{equation}
\pi \left( \bm{\beta}_i, \sigma \right) \propto \sigma^{-2}.
\end{equation}

Now, let us denote the sampling distribution by $f(\bm{y}_i,\sigma|\bm{\beta}_i)$, then the joint density of $\bm{y}_i$, $\sigma$ and $\bm{\beta}_i$ is given by:
$
f(\bm{y}_i,\bm{\beta}_i,\sigma) = f(\bm{y}_i,\sigma|\bm{\beta}_i) \pi (\bm{\beta}_i, \sigma).
$
The marginal distribution of $\bm{y}_i$ is then given by:
$
m(\bm{y}_i) = \iint f(\bm{y}_i,\sigma|\bm{\beta}_i) \pi (\bm{\beta}_i, \sigma) d\bm{\beta}_i d\sigma
$
To obtain the posterior distribution of $\bm{\beta}_i$, we calculate the conditional distribution using the Bayes' Rule:
$
\pi \left( \bm{\beta}_i| \sigma, \bm{y}_i \right) = \frac{f(\bm{y}_i,\bm{\beta}_i,\sigma)}{m(\bm{y}_i)} = \frac{ f(\bm{y}_i|\sigma, \bm{\beta}_i) \pi (\bm{\beta}_i,\sigma)}{\iint f(\bm{y}_i,\sigma|\bm{\beta}_i) \pi (\bm{\beta}_i) d\bm{\beta}_i d\sigma}
$ 

The conditional posterior of $\bm{\beta}_i$, given $\sigma$ is then:
\begin{equation}
\pi \left( \bm{\beta}_i|\sigma ,\bm{y}_i \right) 
\sim N\left( \bm{\hat{\beta}}_i, 
\left( \bm{X}^T\bm{X} \right)^{-1} \sigma^2 \right);
\end{equation}
where $\bm{\hat{\beta}}_i = \left( \bm{X}^T\bm{X} \right)^{-1}\bm{X}^T \bm{y}_i$.

The marginal posterior of $\sigma^2$ can be obtained by factoring the 
joint posterior distribution of $\bm{\beta}_i$ and $\sigma^2$ as:
$
\pi (\sigma^2|\bm{y}_i) = \frac{\pi \left( \bm{\beta}_i,\sigma^2 
|\bm{y}_i \right)}{\pi \left( \bm{\beta}_i|\sigma^2 ,\bm{y}_i \right)};
$
$
\pi (\sigma^2|\bm{y}_i) \sim Inv- \chi^2(n-k,s^2).
$
Where $n$ is the sample size and $k$ is the number of parameters in the model, and:
$
s^2 = \frac{1}{n-k} \left( \bm{y}_i - \bm{X}\bm{\hat{\beta}}_i \right)^T
 \left( \bm{y}_i - \bm{X}\bm{\hat{\beta}}_i \right).
$

Finally, for each voxel, $i$, in the region of interest of our study, 
we calculate the posterior probability that the coefficient associated with the 
stimulus, $t$, is not zero, which is 
$P(\bm{\beta}_{i,t} > 0 | \bm{y}_i, \bm{X}, \sigma)$.

Let $\bm{\mathbb{P}} = \left\{ P(\bm{\beta}_{i,t} > 0 | \bm{y}_i, \bm{X},\sigma) \right\}_{i=[1,v]}$ 
represent a \gls{ppm}, where $v$ is the number of voxels in the region of interest of a \gls{fmri} experiment. 
Our goal now is to calculate a threshold and find activated regions using $\bm{\mathbb{P}}$, 
for which we propose the \gls{bfast} algorithm.

\section{\texorpdfstring{\gls{bfast}}{BFAST} Algorithm}

\subsection{\texorpdfstring{\acrlong{tn}}{Truncated Nornal} Distribution}

All the entries of $\bm{\mathbb{P}}$ are probabilities, hence, we will assume the
$\bm{\mathbb{P}}$ as a \gls{tn} Distribution in the interval $[0,1]$, i.e.:
$
\bm{\mathbb{P}} \sim TN\left( \mu_{\bm{\mathbb{P}}}, \sigma^2_{\bm{\mathbb{P}}}, 0,1 \right).
$
The mean $\mu_{\bm{\mathbb{P}}}$ and the variance $\sigma^2_{\bm{\mathbb{P}}}$ can be 
regarded as a perturbation of the mean $m$ and variance $\tau^2$ of the parent normal 
distribution, respectively. Its values can be determined by referencing the normal
 \gls{pdf} $\phi$ and \gls{cdf} $\Phi$. For a \gls{tn} between 0 and 1, we calculate the mean and variance as
presented in \cite{johnson1995continuous}:

With:

\begin{equation}
\alpha = \frac{-m}{\tau}; \quad \beta = \frac{1-m}{\tau}
\label{eq:alpha_beta}
\end{equation}

We have:

\begin{equation} \label{eq:mu_P}
\mu_{\bm{\mathbb{P}}} = m - \tau \cdot \frac{\phi(0,1;\beta)-\phi(0,1;\alpha)}{\Phi(0,1;\beta)-\Phi(0,1;\alpha)}
\end{equation}

And:

\begin{equation} \label{eq:sigma_P}
\sigma^2_{\bm{\mathbb{P}}} = \tau^2 \cdot \left( 1 - \frac{\beta \phi(0,1;\beta)- \alpha \phi(0,1;\alpha)}{\Phi(0,1;\beta)-\Phi(0,1;\alpha)} - \left( \frac{\phi(0,1;\beta)-\phi(0,1;\alpha)}{\Phi(0,1;\beta)-\Phi(0,1;\alpha)} \right)^2 \right)
\end{equation}

\subsection{\texorpdfstring{\acrlong{evt}}{Extreme Value Theory}}

To find the threshold probability value that helps us find active and inactive voxels, 
we will consider the extreme value distribution for the \gls{tn} distribution is 
used \cite{burkardt2014truncated, nadarajah2004beta}. From Theorem 
10.5.2 of \cite{david2004order}, it is deduced that a \gls{tn} distribution is 
in the domain of maximal attraction of a Gumbel distribution ($G$) (Appendix \ref{ap:theoremVer}).

Hence, we can say that $\exists a_v>0$ and $b_v$ and a nondegenerate \gls{cdf} $G$ 
such that $TN^v(a_vx+b_v) \rightarrow G(x)$ at all continuity points of $G$. We can choose:

\begin{equation}
a_v = \left[ v\psi(b_v) \right]^{-1}; \quad b_v = \Psi^{-1}(1-1/v).
\label{eq:av_bv}
\end{equation}

Typically, $\psi$ and $\Psi$ are used as the \gls{pdf} and \gls{cdf} of the \gls{tn}, 
respectively.

\subsection{Gaussian Kernel Smoothing}

The \gls{bfast} algorithm also uses Gaussian Smoothing, a spatial filtering technique 
commonly used in image and signal processing, to enhance images by reducing noise and 
preserving essential features \cite{garg2016quality}. This method applies a Gaussian kernel, 
characterized by its bell-shaped curve, to each pixel in an image or \gls{ppm} in this case. 
The kernel serves as a weighted averaging filter where the central pixel or element has the 
highest weight while the surrounding pixels or elements contribute with decreasing weights as 
their distance from the center increases. The mathematical basis of Gaussian Smoothing lies 
in convolution, where the kernel is convolved with the input data, blurring the image or 
signal. The smoothing degree depends on the Gaussian kernel's standard deviation, $\sigma_s$. 
A larger standard deviation results in more significant smoothing, and a more minor standard 
deviation results in less smoothing.

\subsection{\texorpdfstring{\acrlong{ji}}{Jaccard Index}}

A version of the \gls{ji} is also used in the \gls{bfast} algorithm. We define 
the \gls{ji}, $J(\bm{A},\bm{B})$, as a similarity index between images $\bm{A}$ and $\bm{B}$, 
and its computed as a quotient:

\begin{equation}
J(\bm{A},\bm{B}) = \frac{|\bm{A} \cap \bm{B}|}{|\bm{A} \cup \bm{B}|}
\end{equation}

\subsection{\texorpdfstring{\gls{bfast}}{BFAST} Algorithm}

The proposed algorithm called \gls{bfast} can be described as follows:

\begin{enumerate}
\item \textbf{\textit{Initial Setup.}} Start with a \gls{ppm} 
$\bm{\mathbb{P}^{(0)}} = \bm{\mathbb{P}}$. Assume that all voxels are inactive, 
i.e., $\zeta_i \equiv 0 \forall i$, where $\zeta_i$ is 1 when voxel $i$ is 
activated and 0 otherwise. Set $\zeta_i^{(0)} \equiv \zeta_i$ and $v_0 = v$, where $v_k$ 
denotes the number of voxels for which $\zeta_i^{(k)} = 0$.
\item \textbf{\textit{Iterative Steps,}} For $k=1,2,\dots,$ iterate as follows:
\begin{enumerate}
\item \textit{Smoothing}. Smooth $\bm{\mathbb{P}^{(k-1)}}$ using a Gaussian Kernel 
to obtain $\bm{\mathbb{P}^{(k)}}$. Let $\sigma_s = 0.65 + 100(k-1)$. Note that the smoothing
is symmetric in all directions, i.e., $\sigma_s$ is constant in $x,y$ and $z$.
\item \textit{Thresholding.} This consists of three steps:
\begin{enumerate}
\item Calculate $\mu_{\bm{\mathbb{P}^{(k-1)}}}$ and $\sigma^2_{\bm{\mathbb{P}^{(k-1)}}}$ to 
estimate $\mathbb{P}^{(k-1)}$ as a \gls{tn}. Use Equations (\ref{eq:mu_P}) and 
(\ref{eq:sigma_P}) with $m$ and $\tau^2$ being the 
mean and variance of $\mathbb{P}^{(k-1)}$.
\item Calculate $a_v$ and $b_v$. Use Equations (\ref{eq:av_bv}), with $\psi$ and $\Psi$ as
 the \gls{pdf} and \gls{cdf} of 
 $TN\left( \mu_{\bm{\mathbb{P}^{(k-1)}}}, \sigma^2_{\bm{\mathbb{P}^{(k-1)}}}, 0,1 \right)$, 
 respectively.
\item Calculate the probability threshold, 
$\eta=a_v\iota_{0.01}+b_v$, with $\iota_{0.01}$ be the upper-tail 
$0.01$-value of the standard Gumbel Distribution.
\end{enumerate}
\item \textit{Activation}: Set $\zeta_i^{(k)} = 1$ if $\zeta_i^{(k-1)} = 0$ and the value 
of the $i$th voxel of $\mathbb{P}^{(k)}$ is greater than $\eta$. Finally, 
calculate $v_k=\sum_{i=1}^v\left(1-\zeta_i^{(k)}\right)$.
\end{enumerate}
\item \textbf{\textit{Termination.}}
\begin{enumerate}
\item Declare no activation and terminate if $\bm{\zeta}^{(1)} \equiv 0$.
\item If $J(\bm{\zeta}^{(k)},\bm{\zeta}^{(k-1)}) \geq J(\bm{\zeta}^{(k+1)},\bm{\zeta}^{(k)})$, the algorithm terminates and the final activation map is $\bm{\zeta}^{(k)}$.
\item The maximum number of iterations is by default $k=10$.
\end{enumerate}
\end{enumerate}

% Contestar las preguntas de roberto en párrafo
The \gls{bfast} algorithm encompasses several assumptions and parameter 
choices that warrant further investigation for a comprehensive understanding 
of the methodology. Firstly, it is important to note that the Bayesian analysis 
of the time series primarily considers temporal dependence, with spatial 
dependence only partially addressed through smoothing during each iteration. 
Additionally, some parameters within the algorithm are selected for convenience 
but are subject to modification in future iterations. For instance, the 
initial value and increment for $\sigma_s$ at each iteration were chosen 
arbitrarily to allow minimal initial smoothing, with a progressively greater 
effect in subsequent iterations. Furthermore, the upper-tail 0.01-value of 
the standard Gumbel distribution is currently used to calculate the threshold 
at each iteration; however, this parameter could be adjusted to a 0.05-value 
or be made data-dependent in future versions. Finally, the maximum number of 
iterations is set to 10 to ensure significant smoothing, though this parameter 
can also be modified in future iterations to refine the algorithm's performance.