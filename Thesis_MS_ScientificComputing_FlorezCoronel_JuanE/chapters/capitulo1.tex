\chapter{Introduction}

\section{Motivation and Justification}

\gls{fmri} is a non-invasive neuroimaging technique that measures brain 
activity by detecting changes in blood flow \cite{buchbinder2016functional, 
logothetis2008we, christopher2008applications}. Several \gls{fmri} studies 
explore the brain regions involved in language processing, memory, and 
decision-making \cite{gaillard2003developmental,golby2005memory,heekeren2003fmri}. 
One of the primary objectives in \gls{fmri} is to identify the brain regions 
that are activated in response to specific stimuli or 
task \cite{orchard2003simultaneous, deneux2006using, ardekani1999activation}. 
This objective is challenging to approach in low-signal and small-data scenarios. 
These arise when studying high-level cognitive tasks or single-subject 
experiments, respectively. This process of identifying activation regions 
usually involves comparing the \gls{hrf} during the presentation of a 
stimulus or task to the \gls{hrf} during a resting state or control 
condition \cite{arthurs2002well, logothetis2004nature, lee2013resting}.

% Change "compare" in the third sentence with "measure"/"meson."
\gls{hrf} is a convolution of a discrete variable and some continuous 
function. This function mainly relates to the \gls{bold} response. 
To compare the \gls{bold}, researchers have used time-series analysis, 
statistical parametric mapping, multivariate pattern classification, 
Bayesian modeling, among other methods 
\cite{adrian2018complex, marchini2004comparing, mumford2012deconvolving, makni2008fully}. 
These methods are helpful in different situations, such as analyzing data 
points over time, processing spatially distributed processes, combining 
spatial and temporal patterns, and using probabilistic predictions. 
To some extent, these situations are partially present in some aspects 
of this research; hence, the methods will be helpful.

Researchers also used methods based in \gls{ast} for \gls{fmri} studies. 
In their studies, the problems addressed went from finding the extent and 
shape of the activation region to the identification of the more accurate 
smoothing technique and procedure 
\cite{tabelow2006analyzing, lindquist2010adaptive, strappini2017adaptive,almodovar2019fast}. 
The advantage of the \gls{ast} method as it has been used is the ability to 
estimate thresholds between contrast-based maps and reduce noise inherent to 
the \gls{fmri} experiments. This approach yields good results in accurately 
identifying activated regions in \gls{fmri} experiments. However, exploring 
different methods, such as a Bayesian approach, where probability maps are 
used instead, can result in more precisely identifying such activated regions.

This study proposes a Bayesian approach using \gls{ast} methods for the 
activation detection problem in single-subject \gls{fmri}. As opposed to 
previous research where \gls{ast} for \gls{fmri} is used, in a Bayesian 
approach, the smoothing procedure will occur in the probability maps, 
resulting in a more understandable interpretation when finding activated 
regions. Although results might not be improved compared to a frequentist 
approach, it is relevant to explore the possible benefits of this kind of 
technique. 

\section{Objectives}

\begin{itemize}
\item Perform Bayesian time-series analysis to obtain a posterior probability 
map of an \gls{fmri} image for a single-subject situation.
\item Develop an \gls{ast} method that inputs the probability posterior map 
and finds the possible activated voxels.
\item Study the proposed algorithm in different simulation frameworks. Study 
the results in terms of similarity, rate of false positives, and percent of 
activation.
\item Finally, apply the algorithm to a real dataset.
\end{itemize}

\section{Intellectual Merit}

The \gls{bfast} algorithm contributes to neuroimaging with a non-existent procedure for 
detecting activation in \gls{fmri} images. By combining Bayesian analysis with 
adaptive smoothing and thresholding, this method effectively addresses the 
challenges of low-signal and single-subject situations. This innovative 
approach improves the accuracy and reliability of activation detection. 

\section{Broader Impacts}

The algorithm's development holds significant promise, especially in 
cognitive studies. Enhanced accuracy in \gls{fmri} activation detection will 
aid research in cognitive psychology, neuroscience, and related fields, 
making it a valuable tool for investigating individual cognitive processes 
and brain-behavior relationships. Sharing this research through publications 
and conferences will promote interdisciplinary collaboration and advance 
scientific knowledge. Additionally, this work has educational benefits, 
providing a basis for training future researchers in advanced neuroimaging 
techniques and Bayesian analysis methods.

\section{Chapter Summary}

This thesis consists of 7 chapters, which are briefly summarized below:
\begin{description}
\item [Chapter 1: Introduction.] 
This chapter introduces the research project and provides an overview of its objectives, significance, and scope. Topics such as \gls{fmri}, \gls{bold}, and \gls{ast} are briefly explained.
\item [Chapter 2: Literature Review.]
Chapter 2 presents a comprehensive review of the relevant literature on the study. In this chapter, \gls{fmri} studies, alongside statistical theory and models, are deeply discussed. The chapter highlights the existing gaps and areas where the current study adds value.
\item [Chapter 3: Methodology.]
In Chapter 3, the methodology used in our work is described. The chapter details the design and development of the models and algorithms proposed. The methodology is clearly described, ensuring the study's replicability.
\item [Chapter 4: Experimental Simulation.] 
Chapter 4 defines the creation and analysis of simulated data where the ground truth is known. The structure proposed for the simulation enables an evaluation of the accuracy of our methods. By using simulated data, we aim to validate and understand the capabilities of our approach.
\item [Chapter 5: Performance Evaluation Results.]
In Chapter 5, the proposed algorithm's results applied to experimental simulation data are presented. The chapter summarizes the evaluation of our work in different scenarios of signal magnitudes. The results presented in this chapter ensure that the results from the next chapter are accurate.
\item [Chapter 6: \acrshort{bfast} in a Real Dataset]
This chapter presents the results of the proposed algorithm in real datasets of \gls{fmri} experiments. The objective of this chapter is to show an example of the usage of our work. In the example experiment, the processing of beliefs and questions is taken as stimuli.
\item [Chapter 7: Conclusion and Future Work.]
In Chapter 7, the study concludes by summarizing the main findings and implications. This chapter also reflects on the study's limitations and identifies areas for improvement. The chapter serves as a closing remark, providing a comprehensive research summary and emphasizing its contributions to the field.
\end{description}