\chapter{\texorpdfstring{\gls{3d}}{3D} True Maps Generation}
\label{ap:3dMapsGen}

The procedure used to generate the \gls{3d} maps is explained below
with the following Python functions. First, Listing 
\ref{lst:createImage} presents the \textit{createImage} function,
which is the final step that creates the map, the inputs 
of this function are the dimensions of the map. This function creates a
\gls{3d} space and randomly selects some coordinates inside this space.
Now, note that this function requires the usage of the 
\textit{createActivationCluster} function, shown in 
Listing \ref{lst:createActivationCluster}. The objective of this second
function is to generate activation clusters around the randomly selected
coordinates. This clusters have a random shape within a fixed radious.
Finally, we use the functions \textit{activateLoners} and \textit{surrAct} shown 
in Listings \ref{lst:activateLoners} and \ref{lst:surrAct}, respectively. This 
functions have the task to correct the shape of the previously generated clusters
so they do not have unusual holes in them.

\begin{lstlisting}[language=Python, caption=\textit{createImage} Function, label=lst:createImage]
def createImage(dimX,dimY,dimZ):
    p = (np.random.random((dimX,dimY))>0.99).astype(int)
    base = np.zeros((dimX,dimY,dimZ))
    s = p.shape
    for i in range(s[0]):
      for j in range(s[1]):
        if p[i,j]:
          k = np.random.randint(0,dimZ)
          createActivationCluster(base,i,j,k,2*dimZ)
    return(base)
\end{lstlisting}

\begin{lstlisting}[language=Python, caption=\textit{createActivationCluster} Function, label=lst:createActivationCluster]
def createActivationCluster(array,i,j,k,size):
    r = int(np.round(np.sqrt(size*3/4/np.pi)))
    array[i,j,k] = 1
    for x in range(-r,r+1):
      for y in range(-r,r+1):
        for z in range(-r,r+1):
          if np.linalg.norm([x,y,z]) <= r:
            try:
              array[i+x,j+y,k+z] = int(np.random.randint(3)>0)
            except:
              pass
    activateLoners(array)
\end{lstlisting}

\begin{lstlisting}[language=Python, caption=\textit{activateLoners} Function, label=lst:activateLoners]
def activateLoners(array):
    s = array.shape
    for x in range(s[0]):
      for y in range(s[1]):
        for z in range(s[2]):
          if array[x,y,z] == 0:
            if surrAct(array,x,y,z):
              array[x,y,z] = 1
\end{lstlisting}

\begin{lstlisting}[language=Python, caption=\textit{surrAct} Function, label=lst:surrAct]
def surrAct(array,i,j,k):
    s = 0
    for x in [-1,1]:
      for y in [-1,1]:
        for z in [-1,1]:
          try:
            s += array[i+x,j+y,k+z]
          except:
            pass
    if s >= 4:
      return(True)
    else:
      return(False)
\end{lstlisting}